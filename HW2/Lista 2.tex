\documentclass[12pt]{article}
\usepackage[utf8]{inputenc}
\usepackage{graphicx}
\usepackage{amsmath}
\usepackage{amssymb}
\usepackage{amsfonts}
\usepackage[margin=2.5cm]{geometry}
\usepackage{listings}
\usepackage{dsfont}
\usepackage[svgnames]{xcolor}

\lstset{language=R,
	showspaces=false,
	showstringspaces=false,
	showtabs=false,
	keywordstyle=\color{NavyBlue},
	commentstyle=\color{YellowGreen},
	stringstyle=\color{Crimson}
}

\setlength{\jot}{1.5ex}

\title{Lista 2 de Finanças Quantitativas\\
Diogo Wolff Surdi}

\begin{document}
\maketitle

\section*{Questão 1}

A distribuição da GPD é:

\begin{equation*}
\mathbb{P}(X \leq x)= \begin{cases}
1-(1+\xi\frac{x-\mu}{\sigma})^{-\frac{1}{\xi}}, \; \xi \neq 0\\
1-exp(-\frac{x-\mu}{\sigma}), \; \xi =0
\end{cases}
\end{equation*}
Note que $F_{\mu,\sigma,\xi}(x)$, a c.d.f. da GPD, equivale a $\mu+\sigma X$, onde $X\sim Pareto(\frac{1}{\xi})$. Sabemos que a esperança de uma distribuição de Pareto é $\frac{\alpha}{\alpha-1}$. Com isso, temos que:
\begin{equation*}
\mathbb{E}[\mu+\sigma X]=\mu +\sigma \mathbb{E}[X]=\mu + \sigma \frac{\frac{1}{\xi}}{\frac{1}{\xi}-1}=\mu+\frac{\sigma}{1-\xi}
\end{equation*}
A mean excess function é $e(l)=\mathbb{E}[X-l|X\:\textgreater \;l]$. Temos então que:
\begin{equation*}
e(l)=\frac{\mathbb{E}[(X-l)\mathds{1}_{X\:\textgreater \;l}]}{\mathbb{P}(X\:\textgreater \;l)}=
\frac{\int_{x}^{\infty}{1-F(u)}du}{1-F_{x}}=\frac{\sigma+\xi(u-\mu)}{\xi-1}
\end{equation*}
Essa função é linear em $u$, como esperado.

\section*{Questão 2.2}

\subsection*{1.1}
\begin{equation*}
X \sim Pareto(\frac{1}{\xi}) \Rightarrow F_{X}=
\begin{cases}
	1-(1+\xi x)^{-\frac{1}{\xi}}\; se\; x\:\textgreater \;0\\
	0 \; c.c.
\end{cases}
\end{equation*}
\subsection*{1.2}
$Q=F^{-1}$. Com isso, temos que:
\begin{equation*}
1-(1+\xi Q)^{-\frac{1}{\xi}}=p \Longrightarrow Q=\frac{(1-p)^{-\xi}-1}{\xi}
\end{equation*}
\subsection*{1.3}
Para gerar amostras de $F$, basta-se tomar valores de $Q(U)$.
\subsection*{2}
Para essa distribuição, a probabilidade de se estar em um intervalo é a probabilidade de ele ocorrer em cada distribuição, dado que foi escolhida alguma delas, multiplicada pela probabilidade de se escolher tal distribuição. Com isso, $f_{y}$ é combinação linear das duas distribuições.
\begin{equation*}
f_{y}(y)=\frac{1}{3}\frac{e^{-\frac{y}{2}}}{2}-
\frac{2}{3}\left(1+\frac{y}{2}\right)^{-3}
\end{equation*}
\subsection*{3}
O processo seria o mesmo do item anterior, gerando a CDF de $y$, encontrando sua inversa e então utilizando $F_{y}^{-1}(U)$.

\section*{Questão 2.3}
\subsection*{1.1}
\begin{equation*}
F_{L}(x)=\begin{cases}
	0 & x\;\textless \;0\\
	1-e^{-rx} & x\geq 0
\end{cases}
\end{equation*}
Basicamente, o valor da perda é 0 para um retorno não-negativo, e tem a distribuição dada pela questão para retornos negativos.

\subsection*{1.2}

\section*{Questão 2.4}
\subsection*{1}
O arquivo PCS tem 381 observações, logo geraremos 1905 amostras. Devo observar que o R não estava encontrando a função rpareto, então tive que instalar outro pacote (gPdtest) para gerar as amostras. Esse pacote gera amostras sem o fator de localização m, então somei a estimação dele na amostra.

\begin{lstlisting}[language=R]
library(Rsafd)
library(gPdtest)
PCS.index <- PCS[,2]
PCS.lmom <- gpd.lmom(PCS.index)$param.est

> PCS.lmom
m     lambda         xi 
0.06824332 0.66804349 0.71179240 

PCS.rlmom <- rgp(n=1905, shape=PCS.lmom[3], scale=PCS.lmom[2])
m=PCS.lmom[1]
samplelmom <- PCS.rlmom + m

> head(sample)
[1] 0.3949272 0.8675918 0.4172077 0.6950772 0.7066541 5.0894675
\end{lstlisting}

Acima gerei a amostra com o método de L-momentos. Também irei gerar a amostra com o método de máxima verossimilhança:

\begin{lstlisting}[language=R]
PCS.ml <- gpd.ml(PCS.index)$param.est

> PCS.ml
m    lambda        xi 
0.0700000 0.7130693 0.6335170 

PCS.rml <- rgp(n=1905, shape=PCS.ml[3], scale=PCS.ml[2])
m2=PCS.ml[1]
sampleml <- PCS.ml+m2

> head(sampleml)
[1] 0.2020278 3.0497405 0.8195610 0.3338650 0.8188285 2.0959607
\end{lstlisting}

\subsection*{2}
Nesse item, vou plotar os QQ-plots de ambos os métodos, e tentar ver se algum deles é melhor.

\begin{lstlisting}[language=R]
par(mfrow=c(1,2))

qqplot(samplelmom, PCS.index)+abline(0,1, col='red')
qqplot(sampleml, PCS.index)+abline(0,1, col='red')
\end{lstlisting}

\begin{center}
	\includegraphics*[scale=0.7]{2.png}
\end{center}

Como pode-se ver, ambos têm problemas na estimação para os últimos quantis.

\subsection*{3}
\begin{lstlisting}[language=R]
> ks.test(samplelmom, PCS.index)

Two-sample Kolmogorov-Smirnov test

data:  samplelmom and PCS.index
D = 0.071391, p-value = 0.07861
alternative hypothesis: two-sided
\end{lstlisting}

\section*{Questão 2.6}

\subsection*{1}
\begin{lstlisting}[language=R]
DSPLRet <- diff(log(DSP))

> head(DSPLRet)
[1]  0.007980092 -0.004314643 -0.007344383 -0.003188190 -0.012344792
 -0.006144413
\end{lstlisting}

\subsection*{2}
\begin{lstlisting}[language=R]
MSPLRet <- diff(log(MSP))

> head(MSPLRet)
[1] -0.0016680571  0.0002782028  0.0009267841  0.0004630702  0.0000000000  0.0011104943
\end{lstlisting}

\subsection*{3}

\begin{lstlisting}[language=R]
qqplot(DSPLRet, MSPLRet)+abline(0,1)
\end{lstlisting}

\includegraphics*{3.png}

Pelo formato do Q-Q plot, os retornos diários aparentam ter caudas mais pesadas do que os retornos minuto a minuto.

\subsection*{4}

\begin{lstlisting}[language=R]
> mean(DSPLRet)
[1] 0.0002717871

> var(DSPLRet)
[1] 8.418932e-05

> mean(MSPLRet)
[1] 2.446315e-05

> var(MSPLRet)
[1] 2.4412e-07
\end{lstlisting}

Os valores esperados são parecidos, porém a variância dos retornos diários é muito maior, logo eles provavelmente não vêm da mesma distribuição.

\subsection*{5}
Novamente, o R não está reconhecendo a função fit.gpd do pacote Rsafd. Vou utilizar a função fit.gpd do pacote gld.

\begin{lstlisting}[language=R]
> fit.gpd(DSPLRet)
Region A:
Estimate    Std. Error
alpha    0.0005402   0.0001341
beta     0.0072581   0.0001206
delta    0.4844521   0.0094871
lambda  -0.1590021   0.0106103

Region B:
Estimate    Std. Error
alpha   -0.0005888   0.0003339
beta     0.2720552   0.0034761
delta    0.5112843   0.0036330
lambda   6.1343819   0.0648509

> fit.gpd(MSPLRet)
Region A:
Estimate    Std. Error
alpha   -1.093e-08   4.942e-05
beta     6.388e-04   5.300e-05
delta    5.211e-01   3.873e-02
lambda   9.971e-02   5.149e-02

Region B:
Estimate   Std. Error
alpha   7.162e-05  6.537e-05 
beta    9.745e-03  6.286e-04 
delta   4.868e-01  1.707e-02 
lambda  4.456e+00  2.474e-01 
\end{lstlisting}



\end{document}



