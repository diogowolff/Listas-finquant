\documentclass[12pt]{article}
\usepackage[utf8]{inputenc}
\usepackage{fancyhdr}
\usepackage{graphicx}
\usepackage{amsmath}
\usepackage{amssymb}
\usepackage{amsfonts}
\usepackage[margin=2.5cm]{geometry}
\usepackage{listings}
\usepackage[svgnames]{xcolor}

\lstset{language=R,
	showspaces=false,
	showstringspaces=false,
	showtabs=false,
	keywordstyle=\color{NavyBlue},
	commentstyle=\color{YellowGreen},
	stringstyle=\color{Crimson}
}

\setlength{\jot}{1.5ex}

\title{Lista 5\\
Finanças Quantitativas\\
Diogo Wolff Surdi}

\begin{document}
	
\maketitle

\section*{Questão 4.1}

\subsection*{4.1.1}
O gráfico 1 não pode ser um gráfico de resíduos de uma regressão linear, pois quase todos os resíduos são positivos, exceto um ponto com resíduo próximo a zero. Como a regressão busca minimizar a soma dos quadrados dos resíduos, e a função quadrática é crescente em relação ao módulo de um número, então seria possível diminuir esse valor alterando (para cima) o eixo de resíduos iguais a 0. De modo formal, uma das C.P.O.s da regressão linear é de que a soma dos resíduos é 0, o que claramente não ocorre nesse gráfico.

\subsection*{4.1.2}
A entrada diagonal de índice i da matriz H é medida da distância da observação i em x em relação à observação média em x (no caso multivariado, vetores de médias). Com isso, não é possível inferir informações sobre a diagonal a partir do gráfico dos resíduos, pois não há informações sobre as variáveis regressoras.

\subsection*{4.1.3}
A regressão é uma LAD. Isso pode ser visto devido ao fato de que a linha de regressão se encaixa bem nos pontos "comuns" dos dados, sem aparente influência dos outliers vistos na região superior à direita. O argmin da LAD é a mediana, que é mais robusta do que a média (argmin da regressão de mínimos quadrados), logo o efeito dos outliers é muito menor.

\section*{Questão 2}
Seja $y=X\hat{\beta}$. Nosso objetivo é encontrar
\begin{equation*}
b=\underset{\hat{\beta}_{0}}{argmin}(y-X\hat{\beta})'(y-X\hat{\beta})
\end{equation*}
Sabemos que $(y-X\hat{\beta})'(y-X\hat{\beta})=y'y-\hat{\beta}'X'y-
y'X\hat{\beta}+\hat{\beta}'X'X\hat{\beta}=y'y-2y'X\hat{\beta}+\hat{\beta}'X'X\hat{\beta}$. Derivando em $\hat{\beta}$ encontramos que:
\begin{equation*}
\frac{\partial (y'X\hat{\beta})}{\partial\hat{\beta}}=X'y; \; 
\frac{\partial (\hat{\beta}'X'X\hat{\beta})}{\partial\hat{\beta}}=2X'X\hat{\beta}
\end{equation*}
Igualando a derivada a 0, encotramos que $2X'y=2X'X\hat{\beta}$. Logo
\begin{equation*}
X'y=X'X\hat{\beta} \Longrightarrow \hat{\beta}=(X'X)^{-1}X'y
\end{equation*}

\section*{Questão 3}

\subsection*{3.1}
As ações são:\\
ITUB4
B3SA3
PETR4
BBDC4
PETR3
ABEV3
BBAS3
ITSA4
MGLU3
JBSS3
LREN3
VALE3
SUZB3
GNDI3
WEGE3
NTCO3
RADL3
RAIL3
BBDC3
RENT3
VIVT4
LAME4
UGPA3
EQTL3
BRFS3
BTOW3
BBSE3
CCRO3
BRDT3
SBSP3
GGBR4
KLBN11
SULA11
HYPE3
HAPV3
VVAR3
ENGI11
TIMP3
EGIE3
CSAN3
BPAC11
SANB11
CMIG4
PCAR3
BRML3
COGN3
ELET3
TOTS3
YDUQ3
BRAP4
BRKM5
CRFB3
FLRY3
IRBR3
ELET6
QUAL3
TAEE11
MULT3
CSNA3
MRFG3
CPFE3
AZUL4
EMBR3
ENBR3
MRVE3
CYRE3
CIEL3
GOAU4
BEEF3
IGTA3
USIM5
ECOR3
CVCB3
HGTX3
GOLL4

\subsection*{3.2}
Farei a análise com dois anos de retornos diários, omitindo os dias que não possuem valores ou dos ativos ou da taxa livre de risco.

\begin{lstlisting}[language=R]
library(quantmod)
library(dplyr)

dados <- read.csv('(...)/carteira.csv', stringsAsFactors = FALSE)
codigos <- c(dados$Codigo)
codigos <- paste(codigos, ".SA", sep='')
codigos <- codigos[-77]
codigos <- codigos[-76]
getSymbols(codigos, src='yahoo', from = "2018-05-21", 
	to = "2020-05-21", env = e)

getSymbols('^BVSP', src='yahoo', from = "2018-05-21", 
	to = "2020-05-21", env = e)
\end{lstlisting}

\subsection*{3.3}
Para o código tive que remover a empresa SULA11, pois os dados estavam incompletos (o na.omit acaba cortando informação demais). Ademais, estou utilizando como taxa livre de risco os retornos diários de títulos americanos de 3 meses (não consegui dados bons para títulos brasileiros), acessando o site \\https://www.federalreserve.gov/datadownload/Choose.aspx?rel=H15.
Baixando o arquivo do site acima, o código abaixo deve rodar e retornar um dataframe com as regressões de todas as ações.

\begin{lstlisting}[language=R]
dados2 <- read.csv('(...)/FRB_H15.csv', stringsAsFactors = FALSE)
dados2 <- dados2[-c(1,2,3,4,5),]
dados2$Series.Description <- as.Date(dados2$Series.Description)

riskfree <- with(dados2, dados2[(Series.Description >= "2019-05-21" 
	& Series.Description <= "2020-05-21"), c(1,3)]) 
riskfree <- riskfree[- grep('ND', riskfree$Market.yield.on.U.S...3.month...), ]
riskfree$Market.yield.on.U.S...3.month... <- 
as.numeric(riskfree$Market.yield.on.U.S...3.month...)

riskfree$Market.yield.on.U.S...3.month... <- 
	(1+riskfree$Market.yield.on.U.S...3.month...)^(1/252)-1
riskfree <- xts(riskfree[,-1],order.by=riskfree$Series.Description)

pframe <- do.call(merge, as.list(e))

todos <- data.frame(merge(pframe, riskfree))
todos2 <- todos[, grepl("Close", names(todos))]
todos2 <- cbind(todos2, todos$riskfree)

retornos <- data.frame(matrix(ncol = 77, nrow = 259))
colnames(retornos) <- c('BVSP', codigos, 'riskfree')
retornos$riskfree <- todos2$`todos$riskfree`

for (i in 1:76) {
for (j in 1:259) {
retornos[j,i] = log(todos2[j+1,i]/todos2[j,i])
}

}

retornos <- retornos[,-34]
retornos <- na.omit(retornos)

coefs <- data.frame(matrix(nrow = 74, ncol=3))
colnames(coefs) <- c('empresa', 'alpha', 'beta')
coefs$empresa <- names(retornos)[2:75]

for (i in 1:74){
coefs[i,2] <- coef(lm(retornos[,i+1] - retornos[,76] ~ retornos[,1] -
	 retornos[,76]))['(Intercept)']
coefs[i,3] <- coef(lm(retornos[,i+1] - retornos[,76] ~ retornos[,1] -
	 retornos[,76]))['retornos[, 1]']
}

> head(coefs)
empresa        alpha      beta
1 ITUB4.SA -0.001890671 1.0250159
2 B3SA3.SA -0.001178668 0.9654480
3 PETR4.SA -0.006399874 1.7656781
4 BBDC4.SA -0.003531262 0.9705597
5 PETR3.SA -0.003861232 1.4758013
6 ABEV3.SA -0.002121174 0.8289363

> tail(coefs)
empresa        alpha      beta
69 IGTA3.SA -0.003501721 0.8349665
70 USIM5.SA -0.002212169 0.9357436
71 ECOR3.SA -0.003478726 0.8276687
72 CVCB3.SA -0.002883426 0.6259825
73 HGTX3.SA -0.004570013 0.8144341
74 GOLL4.SA -0.008074353 0.9891407
\end{lstlisting}

\subsection*{3.4}
Para verificar a variação explicada, basta olhar para o $R^2$. Tomando arbitrariamente que uma boa explicação é $R^2\:\textgreater \;0.6$, temos:

\begin{lstlisting}[language=R]
coefs$'R quadrado' <- NA
coefs$'P valor de alpha' <- NA

for (i in 1:74){
coefs[i, 4] <- summary(lm(retornos[,i+1] - retornos[,76] ~ 
	retornos[,1] - retornos[,76]))$r.squared
coefs[i, 5] <- coef(summary(lm(retornos[,i+1] - retornos[,76] ~ 
	retornos[,1] - retornos[,76])))[7]
}

> subset(coefs, coefs$`R quadrado`>0.6)
empresa        alpha      beta R quadrado P valor de alpha
3  PETR4.SA -0.006399874 1.7656781  0.6250968     2.061112e-04
5  PETR3.SA -0.003861232 1.4758013  0.6724419     2.766589e-03
7  BBAS3.SA -0.003676873 1.3266875  0.6726642     1.540962e-03
8  ITSA4.SA -0.006132765 1.7066715  0.7287029     3.817365e-06
10 JBSS3.SA -0.003421278 1.5183309  0.6063160     2.480604e-02
29 BRDT3.SA -0.005286663 0.9207139  0.6264034     9.473536e-09
30 SBSP3.SA -0.004344839 1.3931826  0.6659584     4.665828e-04
45 COGN3.SA -0.006129905 1.7004298  0.7417666     1.719574e-06
52 FLRY3.SA -0.004915613 1.0194310  0.6405466     5.075046e-07
53 IRBR3.SA -0.003455530 0.9219265  0.6241863     1.302938e-04
58 CSNA3.SA -0.003008471 1.2406074  0.6814542     4.582487e-03
\end{lstlisting}
Para testar a hipótese de que $\alpha=0$, basta observar o p-valor do summary(lm). Testando contra o nível de significância 0.025 temos:

\begin{lstlisting}[language=R]
> subset(coefs, coefs$`P valor de alpha`>0.025)
empresa         alpha      beta R quadrado P valor de alpha
1   ITUB4.SA -1.890671e-03 1.0250159  0.5968030       0.07098985
2   B3SA3.SA -1.178668e-03 0.9654480  0.2154303       0.60527442
4   BBDC4.SA -3.531262e-03 0.9705597  0.3063904       0.05187021
6   ABEV3.SA -2.121174e-03 0.8289363  0.3586675       0.12318791
9   MGLU3.SA  4.393194e-05 1.2445474  0.4866509       0.97784039
14  GNDI3.SA -1.967596e-03 1.2205791  0.3616754       0.32739923
19  BBDC3.SA -2.310543e-03 0.8796261  0.4037981       0.08188959
20  RENT3.SA -2.737210e-03 1.0979733  0.5393931       0.03028991
21  VIVT4.SA  1.911944e-03 1.6511584  0.4749589       0.37394520
22  LAME4.SA -3.260190e-03 1.2646892  0.3359930       0.13952544
23  UGPA3.SA -1.444341e-03 0.7598385  0.3281028       0.28379706
24  EQTL3.SA -8.621131e-04 0.9927723  0.3653597       0.59452409
26  BTOW3.SA  4.046280e-04 0.8944781  0.4347814       0.74851594
28  CCRO3.SA -2.496720e-03 1.1756139  0.5872175       0.04179526
32 KLBN11.SA -2.590178e-03 0.7985194  0.3643142       0.04838198
34  HAPV3.SA -1.706038e-03 0.5375073  0.2860372       0.10570440
36 ENGI11.SA -2.426285e-03 0.9491199  0.4508906       0.06243007
43  PCAR3.SA -3.683575e-03 1.2770299  0.4837428       0.02502130
44  BRML3.SA -1.981059e-03 1.3484000  0.4154259       0.31751319
47  TOTS3.SA -2.004734e-03 0.7724116  0.5060994       0.03482618
50  BRKM5.SA -3.908970e-03 1.5341825  0.4789686       0.04954330
54  ELET6.SA -3.250131e-03 0.7649283  0.2960232       0.02697724
59  MRFG3.SA -1.114291e-03 1.1665254  0.5985997       0.34640098
61  AZUL4.SA -1.111553e-03 0.4150506  0.1332584       0.39653016
63  ENBR3.SA -1.229449e-03 1.2350516  0.5443774       0.37969559
64  MRVE3.SA -2.991334e-03 0.9687313  0.2699026       0.13038512
70  USIM5.SA -2.212169e-03 0.9357436  0.4103314       0.11222059
73  HGTX3.SA -4.570013e-03 0.8144341  0.1961488       0.02608367
\end{lstlisting}
Logo apenas algumas empresas respeitam essa parte do modelo.\\\\
Testarei a normalidade do erro usando o teste K-S. Como são cerca de 400 observações, o ponto crítico ao nível de significância 0.05 é em torno de 0.07
\begin{lstlisting}[language=R]
coefs$'K-S test' <- NA

for (i in 1:74){
coefs[i, 6] <- ks.test(resid(lm(retornos[,i+1] - retornos[,76] ~ retornos[,1] - retornos[,76])), "pnorm")
}

> subset(coefs, coefs$`K-S test`<0.07)
[1] empresa          alpha            beta             R quadrado       P valor de alpha K-S test        
<0 rows> (or 0-length row.names)
\end{lstlisting}
Logo os erros não são distribuídos normalmente. \\\\
Para verificar a estabilidade do $\beta$, vou fazer as regressões para o período anterior, entre 2018 e 2019. Devo notar que o quantmod não conseguiu baixar os dados da Natura desse período, e vou cortar essa ação para fazer a comparação.

\begin{lstlisting}[language=R]
coefs2 <- coefs2[-16,]


codigos <- c(dados$Codigo)
codigos <- paste(codigos, ".SA", sep='')
codigos <- codigos[-77]
codigos <- codigos[-76]
codigos <- codigos[-16]
getSymbols(codigos, src='yahoo', from = "2018-05-21", to = "2019-05-21", env = f)

getSymbols('^BVSP', src='yahoo', from = "2018-05-21", to = "2019-05-21", env = f)

dados2 <- read.csv('(...)/FRB_H15.csv', stringsAsFactors = FALSE)
dados2 <- dados2[-c(1,2,3,4,5),]
dados2$Series.Description <- as.Date(dados2$Series.Description)

riskfree2 <- with(dados2, dados2[(Series.Description >= "2018-05-21" & Series.Description <= "2019-05-21"), c(1,3)]) 
riskfree2 <- riskfree2[- grep('ND', riskfree2$Market.yield.on.U.S..Treasury.securities.at.3.month...constant.maturity..quoted.on.investment.basis), ]
riskfree2$Market.yield.on.U.S..Treasury.securities.at.3.month...constant.maturity..quoted.on.investment.basis <- 
as.numeric(riskfree2$Market.yield.on.U.S..Treasury.securities.at.3.month...constant.maturity..quoted.on.investment.basis)

riskfree2$Market.yield.on.U.S..Treasury.securities.at.3.month...constant.maturity..quoted.on.investment.basis <- 
(1+riskfree2$Market.yield.on.U.S..Treasury.securities.at.3.month...constant.maturity..quoted.on.investment.basis)^(1/252)-1
riskfree2 <- xts(riskfree2[,-1],order.by=riskfree2$Series.Description)

pframe <- do.call(merge, as.list(f))

todos <- data.frame(merge(pframe, riskfree2))
todos2 <- todos[, grepl("Close", names(todos))]
todos2 <- cbind(todos2, todos$riskfree2)

retornos <- data.frame(matrix(ncol = 76, nrow = 259))
colnames(retornos) <- c('BVSP', codigos, 'riskfree2')
retornos$riskfree2 <- todos2$`todos$riskfree2`

for (i in 1:75) {
for (j in 1:259) {
retornos[j,i] = log(todos2[j+1,i]/todos2[j,i])
}

}

retornos <- retornos[,-7]
retornos <- na.omit(retornos)

coefs2 <- data.frame(matrix(nrow = 74, ncol=3))
colnames(coefs2) <- c('empresa', 'alpha', 'beta')
coefs2$empresa <- names(retornos)[2:75]

for (i in 1:73){
coefs2[i,2] <- coef(lm(retornos[,i+1] - retornos[,75] ~ retornos[,1] - retornos[,75]))['(Intercept)']
coefs2[i,3] <- coef(lm(retornos[,i+1] - retornos[,75] ~ retornos[,1] - retornos[,75]))['retornos[, 1]']
}

> head(coefs2)
empresa        alpha      beta
1 ITUB4.SA -0.005382903 0.8896331
2 B3SA3.SA -0.005822410 1.1001610
3 PETR4.SA -0.002843259 1.2858247
4 BBDC4.SA -0.004185322 1.0461081
5 PETR3.SA -0.006150539 1.0956626
6 BBAS3.SA -0.004283986 0.6353060

> tail(coefs2)
empresa        alpha      beta
69  USIM5.SA -0.005882214 1.2188308
70  ECOR3.SA -0.004451138 0.9900901
71  CVCB3.SA -0.003573700 0.2644555
72  HGTX3.SA -0.009525867 0.7613893
73  GOLL4.SA -0.005646744 1.2751726
\end{lstlisting} 

Como pode-se ver, os valores de $\beta$ são bem diferentes dos valores de 2019-2020, logo não vale a consistência de $\beta$.

\end{document}



