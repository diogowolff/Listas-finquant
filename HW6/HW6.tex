\documentclass[12pt]{article}
\usepackage[utf8]{inputenc}
\usepackage{fancyhdr}
\usepackage{graphicx}
\usepackage{amsmath}
\usepackage{amssymb}
\usepackage{amsfonts}
\usepackage[margin=2.5cm]{geometry}
\usepackage{listings}
\usepackage[svgnames]{xcolor}

\lstset{language=R,
	showspaces=false,
	showstringspaces=false,
	showtabs=false,
	keywordstyle=\color{NavyBlue},
	commentstyle=\color{YellowGreen},
	stringstyle=\color{Crimson}
}

\setlength{\jot}{1.5ex}

\title{Lista 6\\
Finanças Quantitativas\\
Diogo Wolff Surdi}

\begin{document}
\maketitle

\section*{Questão 6.6}

\subsection*{6.6.1}
Para o modelo (i) temos:
\begin{equation*}
X_{t}-BX_{t}=W_{t}-1.5BW_{t} \implies X_{t}=(1-B)^{-1}(1-1.5B)W_{t}
\end{equation*}
Os polinômios são então $\phi(z)=1-z$ e $\theta(z)=1-1.5z$.\\\\
Para o modelo (ii) temos:
\begin{equation*}
X_{t}-0.8BX_{t}=W_{t}-0.5BW_{t} \implies X_{t}=(1-0.8B)^{-1}(1-0.5B)W_{t}
\end{equation*}
E os polinômios são $\phi(z)=1-0.8z$ e $\theta(z)=1-0.5z$.

\subsection*{6.6.2}
Seguindo o critério do livro para verificar estabilidade e estacionaridade de processos ARMA, verificamos que a raiz de $\phi$ do processo (i) é igual a 1, enquanto que a raiz de $\phi$ do processo (ii) é maior que 1, logo apenas o processo (ii) é estacionário. Do mesmo modo, temos que $\theta$ de (i) é menor que 1 (em módulo) e $\theta$ de (ii) é maior do que 1, logo apenas o processo (ii) é invertível.

\section*{Questão 6.11}

\subsection*{6.11.1}
\begin{lstlisting}[language=R]
sample <- rnorm(5000)

process <- c(1:5000)
process[1] <- sample[1]
for (i in (2:5000)){
process[i] <- process[i-1]+sample[i]
}
\end{lstlisting}
Nesse código, temos que process é a amostra da variável $X_{t}$. \\

\subsection*{6.11.2}
Devo notar que, como tomamos $x_{0}=0$, o primeiro termo acaba se anulando e devo começar a partir do segundo, para evitar dividir 0 por 0.

\begin{lstlisting}[language=R]
esttheta <- c(1:4999)
estsigsq <- c(1:4999)
sum <- 0
sum2 <- 0
sum3 <- 0

for (i in 2:5000){
for (j in 2:i){
sum <- sum + process[j]*process[j-1]
sum2 <- sum2 + process[j-1]^2
}
esttheta[i-1] <- sum/sum2
sum <- 0
sum2 <- 0
}

plot(esttheta)

for (i in 2:5000){
for (j in 2:i){
sum3 <- sum3 + (process[j]-esttheta[j-1]*process[j-1])^2
}
print(sum3)
estsigsq[i-1] <- sum3/(j-1)
sum3 <- 0
}

plot(estsigsq)

DF <- c(1:4999)

for (i in 1:4999){
DF[i] <- (esttheta[i]-1)/sqrt(estsigsq)
}
\end{lstlisting}
O código é bem lento (cheio de fors), mas não encontrei uma solução que fosse (muito) mais eficiente. A estatística de teste converge para 0. O plot das estimativas é:

\begin{center}
\includegraphics*[scale=0.7]{1.png}
\end{center}

\begin{center}
	\includegraphics*[scale=0.7]{2.png}
\end{center}

\subsection*{6.11.3}
\begin{lstlisting}[language=R]
itemc <- matrix(nrow=1000, ncol=500)*0
dummy <- c(1:1000)

for (i in 1:500){
dummy <- rnorm(1000)
itemc[1, i] <- dummy[1]
for(j in 2:1000){
itemc[j, i] <- itemc[j-1, i]+dummy[j]
}
}

XX <- matrix(nrow=999, ncol=500)*0

for (k in 1:500){
for (i in 1:999){
for (l in 2:1000){
for (j in 2:l){
sum <- sum + itemc[j,k]*itemc[j-1,k]
sum2 <- sum2 + itemc[j-1,k]^2
}
XX[l-1, k] <- sum/sum2
sum <- 0
sum2 <- 0
}
}
}

QQ <- matrix(nrow=91, ncol = 15)*0

parcial <- XX[, 1:18]

for (i in 0:89){
QQ[i+1,]<- quantile(XX[99+10*i,], probs = c(0.01, 0.02, 0.03, 0.04, 0.05,
0.1, 0.25, 0.5, 0.75, 0.9, 
0.95, 0.96, 0.97, 0.98, 0.99),
na.rm=TRUE)
}

library(ggplot2)
library(reshape2)

suporte <- data.frame(matrix(nrow=91,ncol=16))
suporte[,2:16]<-QQ
suporte[,1]<-c(1:91)
colnames(suporte) <- c('observacoes', '0.01', '0.02', '0.03', '0.04',
'0.05', '0.1', '0.25', '0.5', '0.75', 
'0.9', '0.95', '0.96', '0.97', '0.98', '0.99')

suporte <- melt(suporte, id.vars='observacoes', variable.name='Quantis')

ggplot(suporte, aes(observacoes,value)) + geom_line(aes(colour = Quantis))
\end{lstlisting}
Como pode ver, esse código é extremamente ineficiente. Após umas 11 horas rodando, consegui gerar a matriz XX, e o resultado está no gráfico do item seguinte. Novamente,tive problemas no primeiro índice, e meu código é entre 99 e 999.

\subsection*{6.11.4}

\begin{center}
	\includegraphics*[scale=0.7]{3.png}
\end{center}

\subsection*{6.11.5}
\begin{lstlisting}[language=R]
cpn <- read.csv('(...)/CPN Historical Data.csv')
cpn$lret <- NA
cpn$lret[1] <- 0
for (i in 2:1258){
cpn$lret[i] <- log(cpn$Price[i]/cpn$Price[i-1])
}

library(tseries)

> adf.test(cpn$Price)

Augmented Dickey-Fuller Test

data:  cpn$Price
Dickey-Fuller = -4.6105, Lag order = 10, p-value = 0.01
alternative hypothesis: stationary

> adf.test(cpn$lret)

Augmented Dickey-Fuller Test

data:  cpn$lret
Dickey-Fuller = -11.647, Lag order = 10, p-value = 0.01
alternative hypothesis: stationary
\end{lstlisting}
Como pode-se ver, o p-valor é 0.01 em ambas, então o correto seria rejeitar a hipótese nesse nível (porém não sei se esse valor é arredondado, ele pode ser algo como 0.009, caso onde não rejeitaríamos, por exemplo). A questão fala que os quantis seriam proxy para a estatística, porém não vi onde utilizá-los na questão.

\section*{Questão 6.13}

\subsection*{6.13.1}
Sendo $\mu$ o drift, temos $X_{t-1}=\mu +X_{t-2}+W_{t-1}$. Então, seguindo indutivamente, encontramos que:
\begin{align*}
X_{t}&=\mu+(\mu+X_{t-2}+W_{t-1})+W_{t}\\
&=2\mu+(\mu + X_{t-3}+W+t-2)+W_{t-1}+W_{t}\\
\vdots\\
&=t\mu+X_{t-t}+W_{1}+...+W_{t}\\
&=t\mu +x_{0}+\sum_{i=1}^{t}W_{i}
\end{align*}

\subsection*{6.13.2}
Temos:
\begin{equation*}
\mathbb{E}[X_{t}]=\mathbb{E}[x_{0}+\mu t + \sum_{i=1}^{t}W_{i}] =
\mu t+n\mathbb{E}[W_{t}]=\mu t
\end{equation*}
E também:
\begin{equation*}
Var(X_{t})=\sum_{i=1}^{t}Var(W_{i})=t\sigma^2
\end{equation*}
Seja então $\phi_{X}(t_{1},t_{2})=cov(X_{t1},X_{t2})$. Temos que:
\begin{equation*}
\phi(t,t)=\mathbb{E}[(X_{t}-\mu_t)^2]=Var(X_{t})=t\sigma^2
\end{equation*}
E também:
\begin{align*}
\phi(t_{1},t_{2})&=\mathbb{E}[X_{t1}, X_{t2}]-\mathbb{E}[X_{t1}]\mathbb{E}[X_{t2}]\\
&=\mathbb{E}[X_{t1}, X_{t2}]-\mu t_{1}\mu t_{2}\\
&=\mathbb{E}\left[\left(x_{0}+\mu t_{1} + \sum_{i=1}^{t_1}W_{i}\right)\left(x_{0}+\mu t_{2} + \sum_{i=1}^{t_{2}}W_{i}\right)\right]-\mu^2t_{1}t_{2}
\end{align*}

\subsection*{6.13.3}
O processo $X_{t}$ tem raízes complexas de módulo menor igual a 1, logo não é estacionário.

\subsection*{6.13.4}
Temos que:
\begin{equation*}
X_{t}=x_{0}+\mu t + \sum_{i=1}^{t}W_{i}
\end{equation*}
E:
\begin{equation*}
X_{t-1}=x_{0}+\mu (t-1) + \sum_{i=1}^{t-1}W_{i}
\end{equation*}
Logo ao subtrair uma da outra encontramos:
\begin{equation*}
X_{t}-X_{t-1}=\mu+W_{t}
\end{equation*}

\subsection*{6.13.5}
O processo tem drift, logo não é estacionário.

\subsection*{6.13.6}
\begin{lstlisting}[language=R]
x0 <- 1.5
mu <- 0.5
n <- 128

noise <- rnorm(n)

valores <- c(1:129)
valores[1] <- x0
for (i in 2:n+1){
valores[i] <- valores[i-1]+mu+noise[i]
}
\end{lstlisting}

\begin{center}
	\includegraphics*[scale=0.8]{4.png}
\end{center}

\end{document}



